% WARNING! Typesetting tables can cause memory overflow. Set --extra-mem-top=10000000 in Typesetting or texmf.cnf

\documentclass[a4paper, titlepage]{article}
\usepackage{milsymb, arev, ltablex, makecell, multirow, tikzpagenodes, vhistory, titlesec}
\usepackage[a4paper, margin=1.75cm]{geometry}
\usepackage[colorlinks=true, urlcolor=blue]{hyperref}
\usetikzlibrary{calc}
\newcolumntype{n}{>{\raggedright\arraybackslash}m{5cm}}
\newcolumntype{s}{>{\centering\arraybackslash}m{1.75cm}}
\newcommand\MilSymb{\textbf{\texttt{MilSymb}} }
\renewcommand\theadfont{\bfseries}
\setcounter{secnumdepth}{4}
\setcounter{tocdepth}{4} 
\title{MilSymb}
\author{Damian Crosby}
\begin{document}
\thispagestyle{empty}
\begin{center}
\begin{tikzpicture}[remember picture]

\coordinate (NE) at ($(current page text area.north east)-(1.5, 1.5)$);
\coordinate (NW) at ($(current page text area.north west)-(-1.5, 1.5)$);
\coordinate (SE) at ($(current page text area.south east)-(1.5, -1.5)$);
\coordinate (SW) at ($(current page text area.south west)-(-1.5, -1.5)$);

\MilLand[faction=hostile, echelon=team, main=infantry, scale=2](NE)
\MilAir[faction=friendly, main=military fixed wing, upper=jammer, lower=light, scale=2](NW)
\MilSeaSurface[faction=neutral, main=hazardous material transport ship, lower=fast, scale=2](SE)
\MilActivity[faction=unknown, main=searching, upper=house to house, scale=2](SW)

\MilLand[faction=unknown, echelon=battalion, main=armoured, upper=missile, lower=long range, scale=2]($(NE)!0.33!(NW)$)
\MilEquipment[faction=neutral, main=heavy machine gun, mobility=pack animal, scale=2]($(NE)!0.66!(NW)$)

\MilSpace[faction=hostile, main=military earth observation satellite, upper=low earth orbit, lower=radar, scale=2]($(SE)!0.33!(SW)$)
\MilInstallation[faction=friendly, main=electric power, upper=nuclear energy, scale=2]($(SE)!0.66!(SW)$)

\MilSeaSubsurface[faction=friendly, main=snorkelling submarine, upper=auxiliary, lower=nuclear type 5, scale=2]($(NE)!0.2!(SE)$)
\MilLand[faction=neutral, echelon=platoon, main=supply, supply={2}{4}, scale=2]($(NE)!0.4!(SE)$)
\MilActivity[faction=hostile, main=attempted criminal activity, upper=rape, scale=2]($(NE)!0.6!(SE)$)
\MilEquipment[faction=unknown, main=tank recovery vehicle, mobility=wheeled semi trailer, scale=2]($(NE)!0.8!(SE)$)

\MilMissile[faction=hostile, left=sub surface, right=launched, scale=2]($(NW)!0.2!(SW)$)
\MilInstallation[faction=unknown, main=civilian telecommunications, upper=television, scale=2]($(NW)!0.4!(SW)$)
\MilSpace[faction=friendly, main=civilian space station, upper=geosynchronous orbit, scale=2]($(NW)!0.6!(SW)$)
\MilLand[faction=neutral, echelon=corps, main=armoured, upper=missile, lower=long range, scale=2]($(NW)!0.8!(SW)$)

\pgfresetboundingbox
\path[use as bounding box] (0,0);

\end{tikzpicture}

{\let\newpage\relax\maketitle}
\end{center}

\clearpage

\begin{versionhistory}
\renewcommand \vhAuthorColWidth{6cm}
\vhEntry{1.0}{22/01/2018}{Damian Crosby}{Creation}
\end{versionhistory}

\clearpage

\section*{Acknowledgments}

The author would particularly like to thank the following members of the \TeX\  stack exchange community for their solutions to problems during this package\rq{}s development:

\begin{itemize}
\item \href{https://tex.stackexchange.com/users/39222}{cfr}
\item \href{https://tex.stackexchange.com/users/9335}{Kpym}
\item \href{https://tex.stackexchange.com/users/586}{Torbj\o rn T.}
\item \href{https://tex.stackexchange.com/users/2388}{Ulrike Fischer}
\end{itemize}

\clearpage

\tableofcontents

\clearpage

\section{Introduction}

\subsection{Package Summary}

\subsection{Package Dependencies}

The \MilSymb package uses the following packages as dependancies:

\begin{itemize}
\item \texttt{tikz}
\item \texttt{fontenc}
\item \texttt{fix-cm}
\item \texttt{arevmath}
\item \texttt{marvosym}
\item \texttt{acronym}
\item \texttt{amssymb}
\item \texttt{xifthen}
\item \texttt{xparse}
\end{itemize}

\subsection{Using \MilSymb}

To use \MilSymb in your document, just include \texttt{\textbackslash usepackage\{milsymb\}} in your preamble. \MilSymb has only been tested on \LaTeX, other \TeX\  flavours will probably not work. All \MilSymb symbols must be placed inside a TikZ environment, either as part of an inline \texttt{tikz} command or an \texttt{tikzpicture} enviroment. 

\subsubsection{Package Options}

\section{Symbol Commands}

\subsection{General Command Structure}

The general structure of a \MilSymb command is as follows. Syntax in \textit{italics} is optional:\\

\texttt{\textbackslash command[key, key=value]\textit{(location)(name)\{label\}}}

\begin{itemize}
\item \texttt{command} is the name of the command. All are prefixed with \texttt{Mil}-, and end with \texttt{Air}, \texttt{Missile}, \texttt{Land}, \texttt{Equipment}, \texttt{Installation}, \texttt{SeaSurface}, \texttt{SeaSubsurface}, \texttt{Mine}, \texttt{Space}, \texttt{Debris} and\\ \texttt{Activity}. These mostly correspond to the categories found in \href{https://www.awl.edu.pl/images/en/APP_6_C.pdf}{APP6-(C)}, except for \texttt{Missile}, \texttt{Mine} and \texttt{Debris}, which have been broken off from \texttt{Air}, \texttt{SeaSubsurface} and \texttt{Space} for convenience. The \texttt{OwnShip} command is an exception to this rule, and does not have the \texttt{Mil}- prefix.

\item \texttt{key} and \texttt{key=value} are the options used to build the symbol, such as faction, main/upper/lower glyphs and additional text fields. Keys with no value define boolean switches, such as \texttt{unclear}. Keys with values can have one parameter, such as \texttt{faction}, or two parameters, such as \texttt{speed leader}. In the latter case, the syntax is \texttt{key=\{value1\}\{value2\}}.
\item \texttt{location} is an optional coordinate or coordinate reference to place the symbol. This is generally needed when placing multiple symbols in one \texttt{tikzpicture}.
\item \texttt{name} is an optional reference label that acts just like the \texttt{name} property of a node in TikZ. It exposes standard rectangle node anchors such as \texttt{north} and \texttt{south}, allowing connectors to be drawn between symbols. This is useful when drawing organisation charts and similar (see Example \ref{}).
\item \texttt{label} is an optional text label that is added to the right of the symbol.
\end{itemize}

\subsubsection{Shared Keys}

\subsection{Air Command (\textbf{\texttt{MilAir}})}

\subsubsection{Symbol Tables}

\paragraph{\texttt{Main}}
%\input{manual_scripts/Air_Main_table.tex}

\paragraph{\texttt{Upper}}
%\input{manual_scripts/Air_Upper_table.tex}

\paragraph{\texttt{Lower}}
%\input{manual_scripts/Air_Lower_table.tex}

\subsubsection{Additional Features}

\paragraph{Speed Leader}

\subsection{Missile Command (\textbf{\texttt{MilMissile}})}

\subsubsection{Symbol Tables}

\paragraph{\texttt{Left}}
%\input{manual_scripts/Missile_Left_table.tex}

\paragraph{\texttt{Right}}
%\input{manual_scripts/Missile_Right_table.tex}

\subsubsection{Additional Features}

\paragraph{Speed Leader}

\subsection{Land Command (\textbf{\texttt{MilLand}})}

\subsubsection{Symbol Tables}

\paragraph{\texttt{Main}}
%\input{manual_scripts/Land_Main_table.tex}

\paragraph{\texttt{Upper}}
%\input{manual_scripts/Land_Upper_table.tex}

\paragraph{\texttt{Lower}}
%\input{manual_scripts/Land_Lower_table.tex}

\subsubsection{Text Fields}

\subsubsection{Additional Features}

\paragraph{Position and Movement}

\paragraph{Echelon}

\paragraph{Status}

\paragraph{Headquarters}

\paragraph{Grouping}

\paragraph{Supply Class}

\subsection{Equipment Command (\textbf{\texttt{MilEquipment}})}

\subsubsection{Symbol Tables}

\paragraph{\texttt{Main}}
%\input{manual_scripts/Equipment_Main_table.tex}

\paragraph{\texttt{Mobility}}
%\input{manual_scripts/Equipment_Mobility_table.tex}

\subsection{Installation Command (\textbf{\texttt{MilInstallation}})}

\subsubsection{Symbol Tables}

\paragraph{\texttt{Main}}
%\input{manual_scripts/Installation_Main_table.tex}

\paragraph{\texttt{Upper}}
%\input{manual_scripts/Installation_Upper_table.tex}

\subsection{Sea Surface Command (\textbf{\texttt{MilSeaSurface}})}

\subsubsection{Symbol Tables}

\paragraph{\texttt{Main}}
%\input{manual_scripts/SeaSurface_Main_table.tex}

\paragraph{\texttt{Upper}}
%\input{manual_scripts/SeaSurface_Upper_table.tex}

\paragraph{\texttt{Lower}}
%\input{manual_scripts/SeaSurface_Lower_table.tex}

\subsubsection{Additional Features}

\paragraph{Speed Leader}

\subsection{Own Ship Command (\textbf{\texttt{OwnShip}})}

\subsection{Sea Subsurface Command (\textbf{\texttt{MilSeaSubsurface}})}

\subsubsection{Symbol Tables}

\paragraph{\texttt{Main}}
%\input{manual_scripts/SeaSubsurface_Main_table.tex}

\paragraph{\texttt{Upper}}
%\input{manual_scripts/SeaSubsurface_Upper_table.tex}

\paragraph{\texttt{Lower}}
%\input{manual_scripts/SeaSubsurface_Lower_table.tex}

\subsubsection{Additional Features}

\paragraph{Speed Leader}

\subsection{Sea Mine Command (\textbf{\texttt{MilMine}})}

\subsection{Space Command (\textbf{\texttt{MilSpace}})}

\subsubsection{Symbol Tables}

\paragraph{\texttt{Main}}
%\input{manual_scripts/Space_Main_table.tex}

\paragraph{\texttt{Upper}}
%\input{manual_scripts/Space_Upper_table.tex}

\paragraph{\texttt{Lower}}
%\input{manual_scripts/Space_Lower_table.tex}

\subsubsection{Additional Features}

\paragraph{Speed Leader}

\subsection{Space Debris Command (\textbf{\texttt{MilDebris}})}

\subsection{Activity Command (\textbf{\texttt{MilActivity}})}

\subsubsection{Symbol Tables}

\paragraph{\texttt{Main}}
%\input{manual_scripts/Activity_Main_table.tex}

\paragraph{\texttt{Upper}}
%\input{manual_scripts/Activity_Upper_table.tex}

\section{Examples}

\section{Control Measures}

Control Measures are planned to be included in the next major version of \textbf{\texttt{MilSymb}}. Please see the \href{https://github.com/ralphieraccoon/MilSymb}{GitHub} repository for further information.

\end{document}
