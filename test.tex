\documentclass[tikz,border=5pt]{standalone}
\usepackage{milsymb}
\begin{document}

\begin{tikzpicture}
%\NATOAir[faction=unknown, main=military fixed wing, upper=search and rescue, lower=heavy, track number=TN 177, call sign=CZ178, position and movement=277/16, nation=MUM, additional information=CHEESE]{(8, 6)}

%\NATOLand[faction=friendly, echelon=platoon, task force, status=reinforced, upper=mobile advisor and support, feint or dummy, country indicator=CAN, altitude={Low}{Medium}, movement={2}{(-1.6, 1)}]{(5,1)}
%
%\NATOLand[faction=friendly, main=supply, supply={3}{}, echelon=platoon, task force, status=reinforced and reduced, upper=mobile advisor and support,  headquarters=command group,  altitude={Low}{Medium}, offset={2}{(-1.25, 0.2)}]{(4,7)};

%\NATOAir[faction=friendly, main=military airship]{(0,0)}

%\NATOLand[faction=unknown, supply={1}{2}]{(1,7)}

\begin{landgroup}
\item \NATOLand[faction=friendly, main=above corps support, echelon=team, status=reinforced and reduced, upper=mobile advisor and support, altitude={Low}{Medium}]{(0,0)};
\item \NATOLand[faction=friendly, main=above corps support, echelon=platoon, task force, status=reinforced and reduced, feint or dummy, upper=mobile advisor and support, altitude={Low}{Medium}]{(0,0)};
\end{landgroup}

%\begin{landgroup}
%\item \NATOLand[faction=friendly, main=above corps support, echelon=platoon, task force, status=reinforced and reduced, upper=mobile advisor and support,  altitude={Low}{Medium}]{(0,0)};
%\item \NATOLand[faction=friendly, main=above corps support, echelon=platoon, task force, status=reinforced and reduced, upper=mobile advisor and support,  altitude={Low}{Medium}]{(0,0)};
%\end{landgroup}

%\pic at (0, 0) {NATOSymb equipment/friendly};
%\pic at (0, 0) {NATOSymb equipment/main/booby trap};
%\pic at (0, 0) {NATOSymb equipment/mobility/railroad};


%\draw[gray] (0:0.5) -- (45:0.5) -- (90:0.5) -- (135:0.5) -- (180:0.5) -- (225:0.5) -- (270:0.5) -- (315:0.5) -- cycle; %DEBUG - symbol template
%\begin{scope}
%\clip (0:0.5) -- (45:0.5) -- (90:0.5) -- (135:0.5) -- (180:0.5) -- (225:0.5) -- (270:0.5) -- (315:0.5) -- cycle;
%	\draw[gray] (0.5, 0.2) -- (-0.5, 0.2);
%	\draw[gray] (0.5, -0.2) -- (-0.5, -0.2);
%\end{scope}


\end{tikzpicture}

\end{document}
